\documentclass[10pt, a4paper]{article}

\usepackage[english]{babel}
\usepackage[utf8]{inputenc}
\usepackage{float}
\usepackage[]{amsmath} 
\usepackage{graphicx}

% Define question and answer command
\newcounter{qcounter}
\newcommand{\q}[2]
{
    \textbf{Task \refstepcounter{qcounter} \arabic{qcounter}: #1} \\
    #2
    \par
    \vspace{0.5cm}
} 

\begin{document}

\begin{titlepage}
\centering
{
 \scshape \LARGE 
EL2450 Homework 2
}
\vfill
Andreas Fr\"{o}derberg - 19880730-7577
\par
Martin Favre - 19920130-0010
\end{titlepage}

\section{Rate Monotonic scheduling}
\label{sec:rate_monotonic_scheduling}

\q %1
{
    Explain what Rate Monotonic scheduling means.
}
{
    Rate Monotonic scheduling means that all tasks are given a priority. At te
    beginning of each cycle, the task with the highest priority is run until
    Rate Monotonic scheduling means that all tasks are given a priority. At te
    beginning of each cycle, the task with the highest priority is run until
    completion.
}

\q %2
{
	Are the three tasks schedulable?
}
{
	Calculating the utilization factor U from 
	\begin{equation}
		U = \sum\limits_{i=1}^n \frac{C_i}{T_i}=\frac{6}{20} + \frac{6}{29} +
        \frac{6}{35} = 0.75
	\end{equation}	
	The rules states that if U $<$ 1 the set is schedulable.
}
\q %3
{
    What are the differences in control performance between the different
    pendulums?
}
{
    When using rate monotonic scheduling, the small and medium pendulums are
    asymptotically stable while the large pendulum is not. The control
    performance is shown in Figure~\ref{fig:T3_dmperformance}.
    \begin{figure}[H]
        \centering
        \includegraphics[width=1\linewidth]{../Matlab/images/Task_3_dmperformance.png}
        \caption{Performance of pendulums under rate monotonic scheduling.}
        \label{fig:T3_dmperformance}
    \end{figure}
}
    
    

\end{document}
