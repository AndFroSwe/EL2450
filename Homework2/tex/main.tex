\documentclass[10pt, a4paper]{article}

\usepackage[english]{babel}
\usepackage[utf8]{inputenc}
\usepackage{float}
\usepackage[]{amsmath} 
\usepackage{graphicx}

% Define question and answer command
\newcounter{qcounter}
\newcommand{\q}[2]
{
    \textbf{Task \refstepcounter{qcounter} \arabic{qcounter}: #1} \\
    #2
    \par
    \vspace{0.5cm}
} 

\begin{document}

\begin{titlepage}
\centering
{
 \scshape \LARGE 
EL2450 Homework 2
}
\vfill
Andreas Fr\"{o}derberg - 19880730-7577
\par
Martin Favre - 19920130-0010
\end{titlepage}

\section{Rate Monotonic scheduling}
\label{sec:rate_monotonic_scheduling}

\q %1
{
    Explain what Rate Monotonic scheduling means.
}
{
    In Rate Monotonic scheduling, each task is given a fixed priority based on
    its task rate. The task with the highest priority is then run at all times.
}

\q %2
{
	Are the three tasks schedulable?
}
{
	Calculating the utilization factor U from 
	\begin{equation}
		U = \sum\limits_{i=1}^n \frac{C_i}{T_i}=\frac{6}{20} + \frac{6}{29} +
        \frac{6}{35} = 0.68
	\end{equation}	
    To be schedulable, the condition 
    \begin{equation}
        U \leq n(2^{\frac{1}{n}} - 1),
    \end{equation}
    where $n$ is the number of processes. Calculating for this example with 3
    tasks, the condition becomes
    \begin{equation}
        0.68 \leq 3(2^{\frac{1}{3}} - 1)=0.78
    \end{equation}
    which holds, therefore the taskset is schedulable with RM scheduling.
}
\q %3
{
    What are the differences in control performance between the different
    pendulums?
}
{
    All pendulums are asymptotically stable and have similar control
    performance. The performance is shown in Figure~\ref{fig:T3_dmperformance}.
    \begin{figure}[H]
        \centering
        \includegraphics[width=1\linewidth]{../Matlab/HW2_sources_Windows64bit/images/Task_3_dm_performance.png}
        \caption{Performance of pendulums under rate monotonic scheduling.}
        \label{fig:T3_dmperformance}
    \end{figure}
}
\q %4
{
    Compare against the schedule in the model. Does it match?
}
{
    As can be seen below in Figure~\ref{fig:Task_4_sch_6ms}, the schedules
    match. The tasks are schedulable as stated in q2.
    \begin{figure}[H]
        \centering
        \includegraphics[width=1\linewidth]{../Matlab/HW2_sources_Windows64bit/images/Task_4_sch_6ms.png}
        \caption{Schedule for pendulums when computation time of all is 6 ms.
        Yellow is small pendulum, blue is medium and red is big.}
        \label{fig:Task_4_sch_6ms}
    \end{figure}
}
    
\q %5
{
    Setting the execution time for all three processes to 10ms, what are the
differences with respect to control performance?
}
{
    For this execution time, the CPU utilization factor becomes
    \begin{equation}
		U = \sum\limits_{i=1}^n \frac{C_i}{T_i}=\frac{10}{20} + \frac{10}{29} +
        \frac{10}{35} = 1.1.
    \end{equation}
    The utilization factor is thus larger than one which means that the
    processes is not schedulable. This is seen in
    Figure~\ref{fig:Task_5_dm_performance} where the small pendulum is no
    longer stable because of the fact that the deadlines are missed.
	\begin{figure}[H]
       		\centering
        	\includegraphics[width=1\linewidth]{../Matlab/HW2_sources_Windows64bit/images/Task_5_dm_performance.png}
        	\caption{Performance of pendulums under rate monotonic schedulingi
            with computation time 10 ms for each task.}
        	\label{fig:T5_dmperformance}
    	\end{figure}
}
%% Earliest deadline first
\section{Earliest Deadline First, EDF}
\label{sec:earliest_deadline_first_edf}
% Reset the task counter
\setcounter{qcounter}{0}
\q %1
{
    Explain how EDF works. What are the advantages and disadvantages of the
    approach compared to RM?
}
{
    In EDF, the task that has the closest deadline to the current time is always
    run. The advantage of this is that the processor can be fully utilized if
    the the scheme is schedulable. Another advantage is that tasks will not get
    starved if the processor is overloaded and deadlines are missed since the
    dynamic prioritization will handle allow all tasks to get processing time.
    One Advantage is that it can be difficult to predict which tasks are run
    when since there is no fixed priority.
}
\q %2
{
    Assume the execition time is changed back to 6 ms. Are the tasks schedulable
    with the EDF? Please motivate your answer and costruct part of the
    corresponding schedule manually.
}
{
    EDF is schedulable as long as CPU utilization is lower than 100 \%. The
    utilization factor is the same as the answer q2 from RM, 0.68. Since
    $U=0.68 \leq 1$ holds, the task set is schedulable with EDF.
}



\end{document}
