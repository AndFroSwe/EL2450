%\documentclass[a4paper,twocolumn]{article} % Document type

\documentclass[a4paper,12pt,oneside,onecolumn]{article} % Document type

\usepackage[left=1.0in, right=1.0in, top=1.0in, bottom=1.0in]{geometry}

\ifx\pdfoutput\undefined
    %Use old Latex if PDFLatex does not work
   \usepackage[dvips]{graphicx}% To get graphics working
   \DeclareGraphicsExtensions{.eps} % Encapsulated PostScript
 \else
    %Use PDFLatex
   \usepackage[pdftex]{graphicx}% To get graphics working
   \DeclareGraphicsExtensions{.pdf,.jpg,.png,.mps} % Portable Document Format, Joint Photographic Experts Group, Portable Network Graphics, MetaPost
   \pdfcompresslevel=9
\fi

\usepackage{amsmath,amssymb}   % Contains mathematical symbols
\usepackage[ansinew]{inputenc} % Input encoding, identical to Windows 1252
\usepackage[english]{babel}    % Language
\usepackage[square,numbers]{natbib}     %Nice numbered citations
\usepackage{siunitx}
\bibliographystyle{plainnat}            %Sorted bibliography



\begin{document}               % Begins the document

\title{Homework 3 in EL2450 Hybrid and Embedded Control Systems}
\author{
  Martin Favre \\ 19920130-0010 \\ mfavre@kth.se 
  \and 
  First name2 Last name2 \\ person number \\ email
  \and
  First name3 Last name3 \\ person number \\ email
  \and
  First name4 Last name4 \\ person number \\ email
  \and
  }
%\date{2010-10-10}             % If you want to set the date yourself.

\maketitle                     % Generates the title




%%%%%%%%%%%%%%%%%%%%%%%%%%%%%%%%%%%%%%%%%%%%%%%%%%%%%%%%%%%%%%%%%%%%%%%%%%%%%%%%%%%
% Instructions regarding the report
%%%%%%%%%%%%%%%%%%%%%%%%%%%%%%%%%%%%%%%%%%%%%%%%%%%%%%%%%%%%%%%%%%%%%%%%%%%%%%%%%%%

\section*{Instructions and Help}

\textbf{Please remove this part and the sample references before submitting your homework.}

Read the general homework instructions available on the course homepage before starting to write the report.

Here are some additional guidelines how to write a homework report.
\begin{itemize}
 \item Fill in name and personal number of all group members.
 \item Do not copy the task descriptions and use the structure below.
 \item Do not include code unless the task explicitly states so.
 \item Motivate your answers well and how you derived them, but be concise.
 \item The number of points is not necessarily related to much you need to write for task.
 \item Put references in the end if any.
 \item Do not include plots from the Simulink scope (color on black background) but export the data to Matlab for plotting.
 \item Include graphics directly in the text and not in a Figure environment, as you normally would. That makes it easier to correct the report.
 \item There is plenty of material available how to use Latex. Use a search engine of your choice to learn more.
\end{itemize}

Here are some examples how to use Latex:
\begin{itemize}
 \item An equation with a reference \eqref{eq_sample} to it
 \begin{equation}
  \dot{x} = \frac{3}{4} x. \label{eq_sample}
 \end{equation}
 \item A multi-line equations with a reference to it
 \begin{align*}
  \hat{x} &= x - y\\
  \alpha &= x + \gamma.
 \end{align*}
 \item An equation in text: $\Phi = \int\limits_{0}^{h} \mathrm{e}^{A\tau} d \tau$.
 \item An image
 \begin{center}
  \includegraphics[width = 0.8\textwidth]{Phase_portrait_with_regionmark}
 \end{center}
 \item A table
 
 \begin{tabular}{@{\vrule height 10.5pt depth4pt  width0pt}|c|c|c|c|}
    \hline
     $-2.46$ & $0$ & $-1.73$ & $0$ \\ \hline
     $0$ & $-2.553$ & $0$ & $2.774$ \\ \hline
     $0$ & $6.172$ & $-10$ & $7.333$ \\ \hline
     $1.767$ & $-0.357$ & $5.714$ & $-6.074$ \\ \hline
  \end{tabular}
  \item A citation \cite{Oetiker:2008:TheNotSoShortIntroductiontoLaTeXe}
  \item Display something exactly as it is written: \verb|\frac{1}{2}_|
  \item Basic formating: \textbf{bold}, \textit{italics}, \texttt{typewriter}
\end{itemize}



\section*{Task 1}

Given the equations 
%	\begin{equation} % Didn't really need these here
%		\dot{x} = \frac{R}{2}(u_l+u_r)cos\theta
%	\end{equation}
%	\begin{equation}
%		\dot{y} = \frac{R}{2}(u_l+u_r)sin\theta
%	\end{equation}
%	\begin{equation}
%		\dot{\theta} = \frac{R}{L}(u_r-u_l)
%	\end{equation}

Given the equations 
	\begin{equation}
		u_\omega = \frac{u_r + u_l}{2}
	\end{equation}
	\begin{equation}
		u_\Psi = u_r-u_l.
	\end{equation}
	
If $u_\omega$ and $u_\Psi$ is given $u_r$ and $u_l$ can be calculated as
%	\begin{equation}    %Partial steps
%		u_l = u_r-u_\Psi
%	\end{equation}
%	\begin{equation}
%		u_r = 2u_\omega - u_l
%	\end{equation}
	\begin{equation}
		u_r = u_\omega + \frac{u_\Psi}{2}
	\end{equation}
	\begin{equation}
		u_l =  u_\omega-\frac{u_\Psi}{2}
	\end{equation}
	
\section*{Task 2}

Using the generated data from Forward.csv and Rotate.csv R and L is determined. Data recieved is time $[\mu$$s]$, position x $[m]$, position y $[m]$ and rotation $\theta$$[^{\circ}]$. %#JobbigasteTecknet2016

$\dot{x}$, $\dot{y}$ and $\dot{\theta}$ is calculated discretely. R is determined by the mean value of all $\dot{x}$ and $\dot{y}$ with the equations
	\begin{equation}
		R_1 = \frac{\dot{x}}{cos(\theta)u_\omega}
	\end{equation}
	\begin{equation}
		R_2 = \frac{\dot{y}}{sin(\theta)u_\omega}
	\end{equation}
	\begin{equation}
		R = \frac{R_1 + R_2}{2}. % Övertydligt
	\end{equation}
L is determined by the mean value of all $\dot{\theta}$
	\begin{equation}
		L =  \frac{R}{\dot{\theta}}u_\Phi
	\end{equation}
	

\section*{Task 3}

Solution to the task

\section*{Task 4}

Solution to the task

\section*{Task 5}

Solution to the task

\section*{Task 6}

Solution to the task

\section*{Task 7}

Solution to the task

\section*{Task 8}

Solution to the task

\section*{Task 9}

Solution to the task

\section*{Task 10}

Solution to the task

\section*{Task 11}

Solution to the task

\section*{Task 12}

Solution to the task

\section*{Task 13}

Solution to the task

\section*{Task 14}

Solution to the task

\section*{Task 15}

Solution to the task

\section*{Task 16}

Solution to the task

\section*{Task 17}

Solution to the task

\section*{Task 18}

Solution to the task

\section*{Task 19}

Solution to the task

\section*{Task 20}

Solution to the task

\section*{Task 21}

Solution to the task

\section*{Task 22}

Solution to the task



%%%%%%%%%%%%%%%%%%%%%%%%%%%%%%%%%%%%%%%%%%%%%%%%%%%%%%%%%%%%%%%%%%%%%%%%%%%%%%%%%%%
% The bibliography
%%%%%%%%%%%%%%%%%%%%%%%%%%%%%%%%%%%%%%%%%%%%%%%%%%%%%%%%%%%%%%%%%%%%%%%%%%%%%%%%%%%
%\bibliography{Bibliography_template} %Read the bibliography from a separate file

\begin{thebibliography}{99}
\bibitem[Khalil(2002)]{Khalil:2002:Nonlinear-systems:vh}
Hassan~K Khalil.
\newblock \emph{Nonlinear systems}.
\newblock Prentice Hall, Upper Saddle river, 3. edition, 2002.
\newblock ISBN 0-13-067389-7.

\bibitem[Oetiker et~al.(2008)Oetiker, Partl, Hyna, and
  Schlegl]{Oetiker:2008:TheNotSoShortIntroductiontoLaTeXe}
Tobias Oetiker, Hubert Partl, Irene Hyna, and Elisabeth Schlegl.
\newblock \emph{The Not So Short Introduction to \LaTeXe}.
\newblock Oetiker, OETIKER+PARTNER AG, Aarweg 15, 4600 Olten, Switzerland,
  2008.
\newblock http://www.ctan.org/info/lshort/.

\bibitem[Sastry(1999)]{Sastry:1999:Nonlinear-systems:-analysis-stability-and-c%
ontrol:xr}
Shankar Sastry.
\newblock \emph{Nonlinear systems: analysis, stability, and control},
  volume~10.
\newblock Springer, New York, N.Y., 1999.
\newblock ISBN 0-387-98513-1.
\end{thebibliography}


\end{document}      % End of the document
