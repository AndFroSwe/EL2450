%\documentclass[a4paper,twocolumn]{article} % Document type
%       Compiler: pdflatex

\documentclass[a4paper,12pt,oneside,onecolumn]{article} % Document type

\usepackage[left=1.0in, right=1.0in, top=1.0in, bottom=1.0in]{geometry}

\ifx\pdfoutput\undefined
    %Use old Latex if PDFLatex does not work
   \usepackage[dvips]{graphicx}% To get graphics working
   \DeclareGraphicsExtensions{.eps} % Encapsulated PostScript
 \else
    %Use PDFLatex
   \usepackage[pdftex]{graphicx}% To get graphics working
   \DeclareGraphicsExtensions{.pdf,.jpg,.png,.mps} % Portable Document Format, Joint Photographic Experts Group, Portable Network Graphics, MetaPost
   \pdfcompresslevel=9
\fi

\usepackage{amsmath,amssymb}   % Contains mathematical symbols
\usepackage[ansinew]{inputenc} % Input encoding, identical to Windows 1252
\usepackage[english]{babel}    % Language
\usepackage[square,numbers]{natbib}     %Nice numbered citations
\usepackage{siunitx}
\usepackage{graphicx}
\usepackage{float}
\bibliographystyle{plainnat}            %Sorted bibliography



\begin{document}               % Begins the document

\title{Homework 3 in EL2450 Hybrid and Embedded Control Systems}
\author{
  Martin Favre \\ 19920130-0010 \\ mfavre@kth.se 
  \and 
  Adam Lang \\ 19861110-3956 \\ adamlang@kth.se
  \and
  Andreas Fr�derberg \\ 19880730-7577 \\ andfro@kth.se
  \and
  }
%\date{2010-10-10}             % If you want to set the date yourself.

\maketitle                     % Generates the title


\section*{Task 1}


Given the equations 
	\begin{equation}
		u_\omega = \frac{u_r + u_l}{2}
	\end{equation}
	\begin{equation}
		u_\Psi = u_r-u_l,
	\end{equation}
and $u_\omega$ and $u_\Psi$, then $u_r$ and $u_l$ can be calculated as
	\begin{equation}
		u_r = u_\omega + \frac{u_\Psi}{2}
	\end{equation}
	\begin{equation}
		u_l =  u_\omega-\frac{u_\Psi}{2}
	\end{equation}
	
\section*{Task 2}

Using the generated data from Forward.csv and Rotate.csv, it is possible
to determine R and L. The data that is recieved, is time $[\mu$$s]$,
position x $[m]$, position y $[m]$ and rotation $\theta$$[^{\circ}]$.
%#JobbigasteTecknet2016

$\dot{x}$, $\dot{y}$ and $\dot{\theta}$ is calculated discretely. With
$R_1$ and $R_2$ and the input $\dot{x}$ and $\dot{y}$
as,
	\begin{equation}
		R_1 = \frac{\dot{x}}{cos(\theta)u_\omega}
	\end{equation}
	\begin{equation}
		R_2 = \frac{\dot{y}}{sin(\theta)u_\omega}
	\end{equation}
We can calculate $R$ as the mean value,
	\begin{equation}
		R = \frac{R_1 + R_2}{2}. % Övertydligt
	\end{equation}
L is determined simularily by calculating the mean values of from all
$\dot{\theta}$ with,
	\begin{equation}
		L =  \frac{R}{\dot{\theta}}u_\Phi
	\end{equation}
        Calculating these mean values will result in
	\begin{align*}
          &R = 5.2\ \si{\milli\meter} \\
          &L = 25.5\ \si{\milli\meter/\ang{1}}
	\end{align*}

\section*{Task 3}
        Asymptotic stability means that starting points close to the
        equilibrium point will push the system towards the same equilibrium point.
        Since this is the case for our system, we can say that is it
        asymptotically stable.
	Zeno behaviour is defined as an infinite amount of state
        transitions in a finite amount of time. This is not valid in
        this case since 
        \begin{equation}
          \lim_{l \to \infty}\sum\limits_{l=1}^\infty t_i(s_i) \neq k
        \end{equation}
        where $s$ is a state transition, $t(s)$ is the function of the time 
        for each state transitions and k is a constant.
        This can also be seen in Figure \ref{fig:task3_plot}.
        \begin{figure}[H]
        \centering
        \includegraphics[scale=0.5]{../matlab/images/task3_plot.png}
        \caption{Response of rot1 simulink model.}
        \label{fig:task3_plot}
    \end{figure}

\section*{Task 4}
        Using the same definition as Task 3, we can say that the system
        is asymptotically stable and in fact does exhibit Zeno
        behaviour. In figure \ref{fig:task3_plot} the systems state
        transitions approach infinite transitions in a finite amount of time. 
	\begin{figure}[H]
        \centering
        \includegraphics[scale = 0.5]{../matlab/images/task4_plot.png}
        \caption{Response of rot2 simulink model.}
        \label{fig:task4_plot}
    \end{figure}
Even though this behaviour is asymptotically stable, it is not practical since a physical system is limited in time and cannot switch infinitely many times.
\section*{Task 5}

	The system is asymptotically stable, and does not exhibit Zeno behaviour. In Figure~\ref{fig:task5_plot} the systems transitions per time unit reaches a constant value. This is because of the added ZOH block in the model which limits the response time of the system to once per sampling time of the ZOH, in this case 0.01 s.
	\begin{figure}[H]
        \centering
        \includegraphics[scale = 0.5, width=1\linewidth]{../matlab/images/task5_plot.png}
        \caption{Response of rot3 simulink model.}
        \label{fig:task5_plot}
    \end{figure}

\section*{Task 6}

Euler forward is a forward difference, in the $z$-transform written as
\begin{equation}
s = \frac{z-1}{\tau_s},
\end{equation}
where $s$ denotes the Laplace transform and the $z$-transform represents a time shift forward in the discrete domain. Applied to the dynamics of the $x$ variable, one yields
\begin{equation}
\frac{z-1}{\tau_s}x=Ru_\omega cos(\theta)
\end{equation}
which with similar calculations for all state variables yields the complete discrete system dynamics
\begin{equation}
x[k+h] = x[k] + \tau_s R u_\omega cos(\theta [k])	\\
\end{equation}
\begin{equation}
y[k+h] = y[k] + \tau_s R u_\omega sin(\theta [k]) \\
\end{equation}
\begin{equation}
\label{thetadynamics}
\theta[k+h] = \theta [k] + \frac{\tau_s R}{L} u_\psi.
\end{equation}


\section*{Task 7}
 	Given
	\begin{equation}
		\dot{\theta} = \frac{Ru_\Psi}{L}
		 \label{eq:thetadot}
	\end{equation}
	and
	\begin{equation}
		u_\Psi{[k]} = K_{\Psi}({\theta}^R  - {\theta}[k]).
	\end{equation}
	With Euler forward equation (~\ref{eq:thetadot}) gives
	\begin{equation}
		\theta[k+h] = \dot{\theta}[k]\tau_s + \theta[k]
	\end{equation}
	\begin{equation}
		\theta[k+h] = \frac{RK\tau_s({\theta}^R  - {\theta}[k])}{L} + \theta[k]
	\end{equation}
	\begin{equation}
R2 = R2 + R(1,i)^2;		\theta[k+h] =\theta[k](1-\frac{RK\tau_s}{L}) + \frac{RK_p\theta^R\tau_s}{L}
	\end{equation}
The eigenvalues are
	\begin{equation}
		\lambda = \left|{(1-\frac{RK\tau_s}{L})}\right| < 1
	\end{equation}
	which gives that 
	\begin{equation}
		0 < K < \frac{2L}{R\tau_s}
	\end{equation}

\section*{Task 8}
    The simulated task chosen was to rotate from $\theta = 0$ to $\theta = 90$.
    The angular response can be seen in Figure~\ref{fig:task8_angleplot}. The
    position's response can be seen inFigure~\ref{fig:task8_posplot} and shows
    that the position does not change.  \begin{figure}[H]
        \centering
        \includegraphics[scale = 0.5]{../matlab/images/task8_angleplot.png}
        \caption{Simulated rotation resonse.}
        \label{fig:task8_angleplot}
    \end{figure}
    \begin{figure}[H]
        \centering
        \includegraphics[scale = 0.5]{../matlab/images/task8_posplot.png}
        \caption{Simulated position response.}
        \label{fig:task8_posplot}
    \end{figure}
As can be seen, it is possible to maintain the angle at the desired value, as suggested by the fact that the poles of the system can be placed within the unit circle. When there is no translational movement, there are no disturbances in the angular controller, which allows for this exact controll.
\section*{Task 9}
plot(x,y)
 	The equation 
	\begin{equation}
		d_0[k] = K_\omega(cos(\frac{\theta[k]\pi}{180}))*(x_0-x[k]) + (sin(\frac{\theta[k]\pi}{180}))*(y_0-y[k])
	\end{equation}
 	describes how $d_0$ changes with time.
\section*{Task 10}
In Figure~\ref{fig:task10_stepplot} the performance of the controller is shown.
Figure~\ref{fig:task10_stepplot} plot(x,y)shows that the controller is able to maintain
its starting position.  \begin{figure}[H]
        \centering
        \includegraphics[scale = 0.5]{../matlab/images/task10_stepplot.png}
        \caption{Simulated position response when setting start to a different position.}
        \label{fig:task10_stepplot}
    \end{figure}
    
    \begin{figure}[H]
        \centering
        \includegraphics[scale = 0.5]{../matlab/images/task10_stillplot.png}
        \caption{Simulated position response when attempting to stand still.}
        \label{fig:task10_stillplot}
    \end{figure}

\section*{Task 11}
In Figure~\ref{fig:task11_angleplot} the error $\theta^R - \theta[k]$ can be seen. In Figure~\ref{fig:task11_d0plot} $d_0$ can be seen.

\begin{figure}[H]
        \centering
        \includegraphics[scale = 0.5]{../matlab/images/task11_angleplot.png}
        \caption{Simulated angular response.}
        \label{fig:task11_angleplot}
    \end{figure}
    
    \begin{figure}[H]
        \centering
        \includegraphics[scale = 0.5]{../matlab/images/task11_d0plot.png}
        \caption{Simulated position response.}
        \label{fig:task11_d0plot}
    \end{figure}

\section*{Task 12}
 	Given
	\begin{equation}
		u_\omega[k]= K_\omega((cos(\theta_g))(x_g-x[k]) + (sin(\theta_g))(y_g-y[k]))
	\end{equation}
	together with
	\begin{equation}
		x[k+1] = x[k]R\tau_scos(\theta[k])u_\omega
	\end{equation}
	\begin{equation}
		y[k+1] = y[k]R\tau_ssin(\theta[k])u_\omega
	\end{equation}
	gives
	\begin{equation}
		x[k+1] = x[k]R\tau_scos(\theta[k])K_\omega((cos(\theta_g))(x_g-x[k]) + (sin(\theta_g))(y_g-y[k]))
		 \label{eq:xkp1}
	\end{equation}
	\begin{equation}
		y[k+1] = y[k]R\tau_ssin(\theta[k])K_\omega((cos(\theta_g))(x_g-x[k]) + (sin(\theta_g))(y_g-y[k]))
		 \label{eq:ykp1}
	\end{equation}
	Looking at the separate stability criteria in both Equation~\ref{eq:xkp1} and~\ref{eq:ykp1} gives
	\begin{equation}
		\lambda-(1-\tau_sRK_{\omega}cos^2(\theta[k])) = 0
	\end{equation}
	\begin{equation}
		\lambda-(1-\tau_sRK_{\omega}sin^2(\theta[k])) = 0
	\end{equation}
	which in turn gives
	\begin{equation}
		\left| (1-\tau_sRK_{\omega}cos^2(\theta[k])) \right | < 1
	\end{equation}
	\begin{equation}
		\left| (1-\tau_sRK_{\omega}sin^2(\theta[k])) \right | < 1
	\end{equation}
	which in turn gives
	\begin{equation}
		K_\omega < \frac{2}{\tau_sRcos^2\theta[k]}
	\end{equation}
	\begin{equation}
		K_\omega < \frac{2}{\tau_sRsin^2\theta[k]}.
	\end{equation}
  	Assuming the worst possible case for theta gives
	%\begin{equation}
	%	0 < K_\plot(x,y)mega < \frac{2}{\tau_sR\theta[k]}.
	%\end{equation}
	A more effective $K_\omega$ is chosen manually through multiple simulations.
\section*{Task 13}

 In figure \ref{fig:task13_positionplot} the position response can be seen.

\begin{figure}[H]
        \centering
        \includegraphics[scale = 0.5]{../matlab/images/task13_positionplot.png}
        \caption{Simulated position response.}
        \label{fig:task13_positionplot}
    \end{figure}

\section*{Task 14}

With the approximation given,
\begin{equation}
\label{eq:simplification}
d_p[k] = p(\theta[k] - \theta^g),
\end{equation}
one can time apply a time shift of one sampling time $\tau_s$ which will yield
\begin{equation}
\label{eq:timeshift}
d_p[k + \tau_s] = p(\theta[k + \tau_s] - \theta^g).
\end{equation}
Using the controller 
\begin{equation}
u_\psi[k] = K_\psi d_p[k]
\end{equation}
and pluggin into (\ref{thetadynamics}) and (\ref{eq:timeshift}) one gets
\begin{equation}
d_p[k + \tau_s] = \frac{p R K_\psi \tau_s}{L} d_p[k] + p(\theta[k] - \theta^g).
\end{equation}
Using (\ref{eq:simplification}) one yields the expression for the dynamics of the controller,
\begin{equation}
\label{eq:dp}
d_p[k+\tau_s] = \left ( 1 + \frac{p R K_\psi \tau_s}{L} \right ) d_p[k].
\end{equation}
The stability criterion for (\ref{eq:dp}) is given by the equation
\begin{equation}
\left | 1 + \frac{p R K_\psi \tau_s}{L} \right | < 1
\end{equation}
the place the pole inside the unit circle.
Allowing negative values, the values of $K_\psi$ for which the controller is stable are then bound by
\begin{equation}
-\frac{2L}{p R \tau_s} < K_\psi < 0.
\end{equation}
\section*{Task 15}
Looking at the simplification (\ref{eq:simplification}), the value of $p$ works like a proportional value in a P-controller. Thus, a higher $p$ makes the system faster but too large a value of $p$ may render the system unstable or cause overshoot.

\section*{Task 16}

  It is theoretically possible to have a controller that adjusts the robot
correctly if we are in continous time. Since we now are dealing with
descrete time it gets harder to adjust the robot perfectly. We would now
need some type of adjusting, non-uniform quantizer and sampler to get
the precision needed for our type of distances. This would have to
increse quantizing and sampling period the closer to $d_p[k]$ we would
get.

In reallity would this be very hard to do with all the disturbances that
exists in the real world implementation in the robot. The most likely
scenario would be a robot that oscillates around $d_p[k]=0$.

  
\section*{Task 17}

    They differ since they now need to recalculate the angle error when the
angle error gets to large. In Figure~\ref{fig:task17} we can see that
$d_p$ increses when the robot starts translation, it will then need to
correct itself on the move.

\begin{figure}[H]
    \centering
    \includegraphics[width=\textwidth]{../matlab/images/task17.png}
    \caption{Control performance when both controllers is activated}
    \label{fig:task17}
\end{figure}


\section*{Task 18}

The hybrid automation is defined as,

\begin{equation}
    H = (Q,S,Init,f,D,E,G,R).
\end{equation}
We can model the controller with $Q$, the discrete state space $S$,
the continuous state space and initial states as,
\begin{align*}
    &Q=(Rotation,Translation,Goal)=(q_1,q_2,q_3) \\
    &S=\mathbb{R}^3 (x,y,\theta)\\
    &Init = Q \times \{ S \in \mathbb{R}^3 :x_0\wedge y_0 \wedge
    \theta_0\}
\end{align*} 
the vector fields, $f$ as,
\begin{align*}
    &f(q_1,S)=(Ru_{\omega}cos\theta,Ru_{\omega}sing\theta,R/Lu_{\Psi}) \\
    &f(q_2,S)=(Ru_{\omega}cos\theta,Ru_{\omega}sing\theta,R/Lu_{\Psi})\\
    &f(q_3,S)=(0,0,0),
\end{align*}
the domains, $D$ as,
\begin{align*}
    &D(q_1)=\{S\in \mathbb{R}^3:\theta \le r_1; |(x,y)-(x_0,y_0)|\le r_2\} \\
    &D(q_2)=\{S\in \mathbb{R}^3:|(x,y)-(x_g,y_g)| \le r_2\} \\
    &D(q_3)=\{S\in \mathbb{R}^3:|(x,y)-(x_g,y_g)| \le r_2\}
\end{align*}
the edges, $E$ as,
\begin{align*}
    &E = \{(q_1,q_2), (q_2,q_3), (q_3, q_1)\},
\end{align*}
the guards, $G$ as,

\begin{align*}
    &G_1(q_1,q_2)=\{\theta\in\mathbb{R}:\theta\le r_1\} \\
    &G_2(q_2,q_1)=\{\theta\in\mathbb{R}:\theta\le r_2\} \\
    &G_3(q_3,q_1)=\{Reset\}
\end{align*}
and the resets, $R$ as,
\begin{align*}
  &R(q_1,q_2,S)=R(q_2,q_3,S)=R(q_3,q_1,S)=\{S\}.
\end{align*}


\section*{Task 19}
  The state transitions can be seen in figure \ref{fig:task19_trans}
  where state 0 is rotating, state 1 is translation and state 2 is
  standstill. 

\begin{center}
    \begin{figure}[H]
      \centering
      \includegraphics[scale = 0.5]{../matlab/images/task19_trans.png}
      \caption{State transitions for the controller between state 1, 2
      and 3}
      \label{fig:task19_trans}
    \end{figure}
\end{center}

This is when driving 1 $\rightarrow$ 3 $\rightarrow$ 6 $\rightarrow$ 5
$\rightarrow$ 8 $\rightarrow$ 1. This can also be seen in the plot in
figure \ref{fig:task13_positionplot} where the performance of the robot
can be seen.

\begin{center}
    \begin{figure}[H]
      \centering
      \includegraphics[scale=0.5]{../matlab/images/task19_cont.png}
      \caption{Control performance for the robot}
      \label{fig:task19_cont}
    \end{figure}
\end{center}

\section*{Task 21}

The difficulty comes from the delay between the computer and the
robot. There is a certain delay between when the user presses the button
and the robot responds. 
\section*{Task 22}
G
Solution to the task



%%%%%%%%%%%%%%%%%%%%%%%%%%%%%%%%%%%%%%%%%%%%%%%%%%%%%%%%%%%%%%%%%%%%%%%%%%%%%%%%%%%
% The bibliography
%%%%%%%%%%%%%%%%%%%%%%%%%%%%%%%%%%%%%%%%%%%%%%%%%%%%%%%%%%%%%%%%%%%%%%%%%%%%%%%%%%%
%\bibliography{Bibliography_template} %Read the bibliography from a separate file

\begin{thebibliography}{99}
\bibitem[Khalil(2002)]{Khalil:2002:Nonlinear-systems:vh}
Hassan~K Khalil.
\newblock \emph{Nonlinear systems}.
\newblock Prentice Hall, Upper Saddle river, 3. edition, 2002.
\newblock ISBN 0-13-067389-7.

\bibitem[Oetiker et~al.(2008)Oetiker, Partl, Hyna, and
  Schlegl]{Oetiker:2008:TheNotSoShortIntroductiontoLaTeXe}
Tobias Oetiker, Hubert Partl, Irene Hyna, and Elisabeth Schlegl.
\newblock \emph{The Not So Short Introduction to \LaTeXe}.
\newblock Oetiker, OETIKER+PARTNER AG, Aarweg 15, 4600 Olten, Switzerland,
  2008.
\newblock http://www.ctan.org/info/lshort/.

\bibitem[Sastry(1999)]{Sastry:1999:Nonlinear-systems:-analysis-stability-and-c%
ontrol:xr}
Shankar Sastry.
\newblock \emph{Nonlinear systems: analysis, stability, and control},
  volume~10.
\newblock Springer, New York, N.Y., 1999.
\newblock ISBN 0-387-98513-1.
\end{thebibliography}


\end{document}      % End of the document
