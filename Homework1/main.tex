\documentclass[10pt, a4paper]{article}

\usepackage[english]{babel}
\usepackage[utf8]{inputenc}
\usepackage{float}
\usepackage[]{amsmath} 

% Define question and answer command
\newcounter{qcounter}
\newcommand{\q}[2]
{
    \textbf{Q\refstepcounter{qcounter} \arabic{qcounter}: #1} \par
    \textbf{A:} #2
       \par
       \vspace{0.5cm}
} 

\begin{document}

\begin{titlepage}
\centering
{
 \scshape \LARGE 
EL2450 Homework 1
}
\vfill
Andreas Froderberg - 19880730-4577
\par
Martin Favre - 19920130-0010
\end{titlepage}


\q
{
    A gain named Tap exists in the Tank 1 model, what is its function?}
{
    The gain Tap models the openable tap on the upper tank. The value 0 means that it is currently closed. A value of 1 means fully opened.
}
\q
{
    Place the poles to give a following step resonse:
    \begin{itemize}
        \item Rise time less than 6s
        \item Overshoot less than 35\%
        \item Settling-time less than 30s
    \end{itemize}
}
{
    With the parameters set to $\chi=0.5$, $\omega_0=0.2$ and $\xi=0.7$ yields
    a system with risetime of 3.31 s and a settling time 18.8 s and an overshoot
    of 22.6\%.
}
\q
{
    What does the reference signal look like?
}
{
    The signal starts at 40 and recieves a step by 10 at 100 s which sets it to
    50.
}
\q
{
    Use the parameter generator to get values.
}
{
    Done.
}
\q
{
    Use the parameters to get different responses. WHich is best?
}
{
    The parameters are:
    \begin{table}[H]
        \centering
        \caption{Parameter values and performance.}
        \label{tab:pvals}
        \begin{tabular}{|c|c|c|c|c|c|}
            \hline
            $\chi$ & $\zeta$ & $\omega_0$ & $T_r$ & $M$ & $T_s$ \\
            \hline
            0.5 & 0.7 & 0.1 & 6.38 & 6.67 & 38.1 \\
            \hline
            0.5 & 0.7 & 0.2 & 3.31 & 22.6 & 18.8 \\
            \hline
            0.5 & 0.8 & 0.2 & 3.19 & 20.9 & 18.4 \\
            \hline
        \end{tabular}
    \end{table}
    The last parameter configuration works best. It is the fastest though it has
    quite significant overshoot, which is still within the given tolerance.
}
\q
{
    What is the cutoff frequency for the open loop system? How was this derived?
}
{
    The open loop system $G_o=F \cdot G$ and the cutoff frequency is the
    frequency when the amplitude gain is 0 dB. Using the bode plots this
frequency is found to be $\omega_c=0.343 \text{rad/s}$. This is confirmed by the
MatLab command \textit{allmargin(Go)}.
}

\q
{
	A ZOH block is placed after the controller. What is the effect?
}
{
	With a ZOH time at h = 1 the systems shows no difference. \\
	With a ZOH time at h = 5 the system starts oscillating. \\
	With a ZOH time at h = 10 the system is severely affected by the few updates and oscillates strongly to compensate. \\
	With a ZOH time at h = 50 is unstable.
}
\q
{
	Discretize the continuous controller into state space form. Replace the simulink controller block with this new discrete controller. What differences can be seen?
}
{
	Show images here Andreas. Not of yourself please.

	Comparison shows that ZOH method yields overall better results. This may be due to that the continuous system on which the ZOH method is based is more precise. 
}

\q
{
 	What sampling time should be chosen?
}
{
	The crossover frequency of the open-loop system is $\omega_c = 0.34 $ A good sampling frequency is $20\omega_c  = 6.8 rad/s$ \\ 
	This gives a samplingtime of $\frac{2\pi}{20\omega_c} = 0.92 $ seconds
}
\q
{
	How long can the samplingtime be without affecting the performance?
}
{
	With manual checks the maximum samplingtime before the systems starts to strongly oscillate is around 2 to 3 times the recommended sampling time from question 9.
}

\q
{
	Simulate system with samplingtime at 4 seconds and estimate the control performance.
}
{
	The system is asymptotically stable but oscillating.
}
\q
{
	Sample G, make Gd
}
{
	\begin{tabular}{|c|c|}
	\hline 
	$a_1$ &  = 0.092 \\
	\hline
	$a_2$ & = 0.074 \\
	\hline
	\end{tabular}
}
\end{document}
