\documentclass[10pt, a4paper]{article}

\usepackage[english]{babel}
\usepackage[utf8]{inputenc}
\usepackage{float}
\usepackage[]{amsmath} 
\usepackage{graphicx}

% Define question and answer command
\newcounter{qcounter}
\newcommand{\q}[2]
{
    \textbf{Q\refstepcounter{qcounter} \arabic{qcounter}: #1} \par
    \textbf{A:} #2
       \par
       \vspace{0.5cm}
} 

\begin{document}

\begin{titlepage}
\centering
{
 \scshape \LARGE 
EL2450 Homework 1
}
\vfill
Andreas Froderberg - 19880730-4577
\par
Martin Favre - 19920130-0010
\end{titlepage}


\q %1
{
    A gain named Tap exists in the Tank 1 model, what is its function?
}
{
    The gain Tap models the bypass tap from the upper tank to the main tank. 
    The value 0 means that it is currently closed. A value of 1 means fully opened.
}
\q  %2
{
    Edit \textit{pid\_design.m} and fill in the transfer functions for the upper
    and lower tank.
}
{
    The figure below shows the result.
    %\begin{figure}[H]
        %\centering
        %\includegraphics[width=\textwidth]{../images/q2_tfs.png}
    %\end{figure}
}
\q %3
{
    What does the reference signal look like?
}
{
    The signal starts at 40 and recieves a step by 10 at 100 s which sets it to
    50.
}
\q %4
{
    Use the parameter generator to get values and fill in the transfer function
    $F$.
}
{
    Done.
}
\q %5
{
    Use the parameters to get different responses. Which is best?
}
{
    The input parameters and their respecitve system parameters are:
    \begin{table}[H]
        \centering
        \caption{Parameter values and performance.}
        \label{tab:pvals}
        \begin{tabular}{|c|c|c|c|c|c|}
            \hline
            $\chi$ & $\zeta$ & $\omega_0$ & $T_r$ [s] & $M [\%]$ & $T_s$ [s] \\
            \hline
            0.5 & 0.7 & 0.1 & 6.38 & 6.67 & 38.1 \\
            \hline
            0.5 & 0.7 & 0.2 & 3.31 & 22.6 & 18.8 \\
            \hline
            0.5 & 0.8 & 0.2 & 3.19 & 20.9 & 18.4 \\
            \hline
        \end{tabular}
    \end{table}
    The last parameter configuration works best. It is the fastest and even though 
    it has quite significant overshoot, it is still within the given tolerance
    and is thus acceptable.
}
\q %6
{
    What is the cutoff frequency for the open loop system? How was this derived?
}
{
    The open loop system is $G_o=FG$ and the cutoff frequency is the
    frequency when the amplitude gain is 0 dB. Using the bode plots this
    frequency is found to be around $\omega_c=0.35 \text{rad/s}$. This is confirmed by the
    MatLab command \textit{allmargin(Go)} which gives $\omega_c = 0.343 \text{rad/s}$.
}
%%%%%% Part 2 %%%%%%%
\q %7
{
    A ZOH block is placed after the controller. What is the effect?
}
{
    The effect of the system for different ZOH times is shown below.
    \begin{figure}[H]
        \centering
        \includegraphics[width=\textwidth]{../Code/images/h1_samplings.png}
        \caption{Upper tank level for different sampling times.}
    \end{figure}
    \begin{figure}[H]
        \centering
        \includegraphics[width=\textwidth]{../Code/images/h2_samplings.png}
        \caption{Lower tank level for different sampling times.}
    \end{figure}
    \begin{figure}[H]
        \centering
        \includegraphics[width=\textwidth]{../Code/images/pump_samplings.png}
        \caption{Pump input for different sampling times.}
    \end{figure}
    From the figures it can be seen that the stabilility and performance of the
    system decreases with higher samplnig time. For a sampling time of about 1
    second, the system is stable and the performance is smooth. At 3 second
    sampling time, it is oscillating and is on the verge of becoming unstable.
}
\q %8
{
    Discretize the continuous controller into state space form. Replace the
    simulink controller block with this new discrete controller. What differences
    can be seen? 
}
{
    The performance of the sampled system is displayed below.
    \begin{figure}[H]
        \centering
        \includegraphics[width=\textwidth]{../Code/images/h1_samplings_ss.png}
        \caption{Upper tank level for different sampling times.}
    \end{figure}
    \begin{figure}[H]
        \centering
        \includegraphics[width=\textwidth]{../Code/images/h2_samplings_ss.png}
        \caption{Lower tank level for different sampling times.}
    \end{figure}
    \begin{figure}[H]
        \centering
        \includegraphics[width=\textwidth]{../Code/images/pump_samplings_ss.png}
        \caption{Pump input for different sampling times.}
    \end{figure}

    Comparison shows that ZOH method yields overall better results. This may be
    due to that the continuous system on which the ZOH method is based is more
    precise.
}

\q %9
{
     What sampling time should be chosen?
}
{
    The crossover frequency of the open-loop system is $\omega_c = 0.34
    \text{rad/s}$. The thumbrule states that the sampling frequency should be
    approximately $20\omega_c  = 6.8 rad/s$. This gives a samplingtime of
    $\frac{2\pi}{20\omega_c} = 0.92 $ seconds
}
\q %10
{
    How long can the samplingtime be without affecting the performance?
}
{
    With manual checks the maximum samplingtime before the systems starts to
    strongly oscillate is around 2 to 3 times the recommended sampling time from
    question 9.
}

\q
{
    Simulate system with samplingtime at 4 seconds and estimate the control
    performance.
}
{
    The system is asymptotically stable but oscillating.
}
\q
{
    Sample G, make Gd
}
{
    \begin{tabular}{|c|c|}
    \hline 
    $a_1$ &  = 0.092 \\
    \hline
    $a_2$ & = 0.074 \\
    \hline
    \end{tabular}
}
\end{document}
